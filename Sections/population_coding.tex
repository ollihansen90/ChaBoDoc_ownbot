% !TEX root = ../svrhm.tex
\section{Population coding}
%Die 4 Varianten benennen, Formeln angeben, Parameter definieren (Sigma, M) (+Referenzen)
%\textit{Anmerkung: Für eine Verallgemeinerung auf Inputgroesse $d$ muessen hier nochmal die Formeln angepasst werden, da wir in der Bachelorarbeit mit $d=2$ gearbeitet haben.}
%This paper looks at four different methods of population coding in order to compare them. These four methods are described below. 
Across all encodings the input values are assumed to be in the range -1 to 1. 
The resulting encoded vector is a concatenation of the encoded components. The visualizations can be found in the appendix \ref{subsecPlots}.

\subsection{Positional Encoding}
For a multimodal example we explore Positional Encoding (PE) as described in \cite{mildenhall2020nerf} uses a combination of sine and cosine functions with exponential frequencies to encode the given input values:
\begin{align*}
    f(x) = [sin(2^0\pi x),\ldots,sin(2^{L-1}\pi x),cos(2^0\pi x),\ldots,cos(2^{L-1}\pi x)]
\end{align*}
for each input scalar $x$. 

\subsection{Fourier Feature Mapping}
Fourier Feature Mapping (FFM) was introduced in \cite{tancik2020fourier}. In contrast to PE, the components are not encoded individually but instead the input vector is transformed via random linear operation:
\begin{align*}
    f(\textbf{p}) = [a_1 sin(2\pi\textbf{b}_1^T\textbf{p}),\ldots,a_L sin(2\pi\textbf{b}_L^T\textbf{p}),a_1 cos(2\pi\textbf{b}_1^T\textbf{p}),\ldots,a_L cos(2\pi \textbf{b}_L^T\textbf{p})].
\end{align*}

In this paper we use $a_k=1$ for all $k$ in order to get a direct comparison between FFM and other encoding methods. The random vectors $\textbf{b}_i$ are drawn independently from a standard normal distribution.

\subsection{Tent Encoding}
For a unimodal example we explore Tent Encoding (TE) by the authors of \cite{jahrens2020solving} is based on the tent mapping.
    \[t(x) = \begin{cases} x+1 & x \in [-1,0] \\ -x+1 & x \in (0,1]\\ 0 & \text{else} \end{cases}\]
In addition to translations $t(\cdot-a)$, TE uses different slopes $m$ that can be adjusted:
\begin{align*}
    &f(x) =\left[t\left(m\cdot(x-\mu_0)\right), t\left(m\cdot(x-\mu_1)\right),\ldots,t\left(m\cdot(x-\mu_{L-1})\right)\right]\\&\text{with }\mu_i=\frac{2\cdot i}{L-1}-1\quad\forall\ i=0,\ldots,L-1.
\end{align*}

\subsection{Magnitude Encoding}
Magnitude Encoding (ME) as described in \cite{jahrens2020solving} uses Gaussian bell curves $g(x)=\exp(-x^2/(2\sigma^2))$. Each input component is encoded via
\begin{align*}
    &f(x) =\left[g\left(x-\mu_0\right), g\left(x-\mu_1\right),\ldots,g\left(x-\mu_{L-1}\right)\right]\\&\text{with }\mu_i=\frac{2\cdot i}{L-1}-1\quad\forall\ i=0,\ldots,L-1.
\end{align*}