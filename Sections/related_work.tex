% !TEX root = ../svrhm.tex
\section{Related work}
Population coding was successfully used in several publications to increase a network's performance. 
The authors of \cite{mildenhall2020nerf} used Positional Encoding on a camera pose vector, a five dimensional input vector $(x,y,z,\theta, \phi)$ with spacial coordinates $x$, $y$ and $z$ as well as orientation $\theta$ and $\phi$ to learn the corresponding RGB-values on the line of sight.
Encoding the low-dimensional inputs was essential to achieve good results on their regression problem. 

In \cite{jahrens2020solving} another population coding scheme called Magnitude Encoding was employed to get a new representation of the individual intensity values of grey scale images. 
The model processes multiple images per sample and uses late fusion, so the improved performance may be attributed to improved information retention.
The input encoding turned out to be crucial to solve certain tasks pertaining to color. 

% In a work focused on optimizing the encoding of coordinates Fourier Feature Mapping \cite{tancik2020fourier} was introduced. 
% The authors have shown the high impact of their encoding scheme on the capacity of MLPs to learn high frequency dependencies between input and output. 
The authors of \cite{tancik2020fourier} introduced Fourier Feature Mapping and demonstated the high impact of their encoding scheme on the capacity of MLPs to learn high frequency dependencies between input and output. 

% Each of these approaches lead to huge improvements in the networks' comprehension of (scalar) data.